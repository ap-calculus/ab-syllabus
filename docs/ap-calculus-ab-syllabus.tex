\documentclass[12pt,fleqn]{article}
\usepackage{lmodern}
\usepackage{amssymb,amsmath}
\usepackage{ifxetex,ifluatex}
\usepackage{fixltx2e} % provides \textsubscript
\ifnum 0\ifxetex 1\fi\ifluatex 1\fi=0 % if pdftex
  \usepackage[T1]{fontenc}
  \usepackage[utf8]{inputenc}
\else % if luatex or xelatex
  \ifxetex
    \usepackage{mathspec}
  \else
    \usepackage{fontspec}
  \fi
  \defaultfontfeatures{Ligatures=TeX,Scale=MatchLowercase}
\fi
% use upquote if available, for straight quotes in verbatim environments
\IfFileExists{upquote.sty}{\usepackage{upquote}}{}
% use microtype if available
\IfFileExists{microtype.sty}{%
\usepackage{microtype}
\UseMicrotypeSet[protrusion]{basicmath} % disable protrusion for tt fonts
}{}
\usepackage[margin=1in]{geometry}
\usepackage{hyperref}
\hypersetup{unicode=true,
            pdftitle={AP® Calculus AB Syllabus},
            pdfauthor={Mr.~Li},
            pdfborder={0 0 0},
            breaklinks=true}
\urlstyle{same}  % don't use monospace font for urls
\usepackage{longtable,booktabs}
\usepackage{graphicx,grffile}
\makeatletter
\def\maxwidth{\ifdim\Gin@nat@width>\linewidth\linewidth\else\Gin@nat@width\fi}
\def\maxheight{\ifdim\Gin@nat@height>\textheight\textheight\else\Gin@nat@height\fi}
\makeatother
% Scale images if necessary, so that they will not overflow the page
% margins by default, and it is still possible to overwrite the defaults
% using explicit options in \includegraphics[width, height, ...]{}
\setkeys{Gin}{width=\maxwidth,height=\maxheight,keepaspectratio}
\IfFileExists{parskip.sty}{%
\usepackage{parskip}
}{% else
\setlength{\parindent}{0pt}
\setlength{\parskip}{6pt plus 2pt minus 1pt}
}
\setlength{\emergencystretch}{3em}  % prevent overfull lines
\providecommand{\tightlist}{%
  \setlength{\itemsep}{0pt}\setlength{\parskip}{0pt}}
\setcounter{secnumdepth}{5}

%%% Use protect on footnotes to avoid problems with footnotes in titles
\let\rmarkdownfootnote\footnote%
\def\footnote{\protect\rmarkdownfootnote}

%%% Change title format to be more compact
\usepackage{titling}

% Create subtitle command for use in maketitle
\providecommand{\subtitle}[1]{
  \posttitle{
    \begin{center}\large#1\end{center}
    }
}

\setlength{\droptitle}{-2em}

  \title{AP\textsuperscript{®} Calculus AB Syllabus}
    \pretitle{\vspace{\droptitle}\centering\huge}
  \posttitle{\par}
    \author{Mr.~Li}
    \preauthor{\centering\large\emph}
  \postauthor{\par}
      \predate{\centering\large\emph}
  \postdate{\par}
    \date{2019-07-04}

\usepackage{booktabs}
\usepackage{amsthm}
\makeatletter
\def\thm@space@setup{%
  \thm@preskip=8pt plus 2pt minus 4pt
  \thm@postskip=\thm@preskip
}
\makeatother

%Allow for more levels in bullet lists
%https://github.com/Witiko/markdown/issues/2
%https://tex.stackexchange.com/questions/41408/a-five-level-deep-list
\usepackage{enumitem}
\setlistdepth{20}
\renewlist{itemize}{itemize}{20}
% initially, use dots for all levels
\setlist[itemize]{label=$\cdot$}

% customize the first 3 levels
\setlist[itemize,1]{label=\textbullet}
\setlist[itemize,2]{label=--}
\setlist[itemize,3]{label=*}

\usepackage{multicol}

\usepackage{bibentry}

% \setcounter{tocdepth}{0}
% \setcounter{secnumdepth}{0}

%YAML in index.Rmd: subparagraph: yes
%https://stackoverflow.com/questions/42916124/not-able-to-use-titlesec-with-markdown-and-pandoc
\usepackage{titlesec}
\titleformat{\section}
   {\normalfont\fontsize{18}{20}\bfseries}{\thesection}{1em}{}
\titleformat{\subsection}
   {\normalfont\fontsize{16}{18}\bfseries}{\thesubsection}{1em}{}
\titleformat{\subsubsection}
   {\normalfont\fontsize{14}{16}\bfseries\sffamily}{}{1em}{}

\usepackage{xcolor}
\usepackage{hyperref}
\hypersetup{
    colorlinks = true,
    linkbordercolor = {white},
    linkcolor = {black},
    citecolor = {black},
    urlcolor = {blue},
    bookmarksopen = false,
    bookmarksdepth=4
}



%https://bookdown.org/yihui/bookdown/latex-index.html
%\usepackage{makeidx}
%\makeindex

\begin{document}
\maketitle

{
\setcounter{tocdepth}{2}
\tableofcontents
}
\hypertarget{course}{%
\section{Course}\label{course}}

\hypertarget{contact}{%
\subsection{Contact}\label{contact}}

\textbf{Teacher}: Mr.~Li

\textbf{E-mail}:

\textbf{Course Website}: \url{https://scholar.netlify.com/calc-ab}

\textbf{Office Hours}: TBD

\begin{itemize}
\tightlist
\item
  You are welcome to stop by anytime during scheduled office hours without an appointment.
\item
  If I need to reschedule office hours, I will send an e-mail to the class.
\item
  If you cannot make it to my office hours, please \href{https://scholar.netlify.com/\#contact}{schedule an appointment}. If none of the time slots work for you, check with me after class so we can arrange a time to meet.
\end{itemize}

\hypertarget{course-description}{%
\subsection{Course Description}\label{course-description}}

\textbf{Prerequisite:} IGCSE Additional Mathematics or the equivalent of Trigonometry/Pre-Calculus

AP\textsuperscript{®} Calculus AB\footnote{AP is a registered trademark of the College Board, which does not endorse and was not involved in the production of this document or any part of this website.} is a one-year course that covers the first semester of college calculus. We will cover all topics covered in the AP Calculus AB exam, including limits, continuity, differentiation, integration, and major theorems and applications of calculus.

\hypertarget{course-objectives}{%
\subsection{Course Objectives}\label{course-objectives}}

\begin{enumerate}
\def\labelenumi{\arabic{enumi}.}
\tightlist
\item
  Develop mastery in univariate calculus concepts.
\item
  Apply differentiation and integration to solve real-world problems.
\item
  Build connections between mathematical proofs and intuition.
\item
  Communicate mathematical ideas clearly in both written and oral form.
\item
  Exercise resilience in independent problem solving.
\item
  Collaborate with classmates in problems involving teamwork.
\end{enumerate}

\hypertarget{resources}{%
\section{Resources}\label{resources}}

\hypertarget{course-textbook}{%
\subsection{Course Textbook}\label{course-textbook}}

Finney, R. L., Demana, F. D., Waits, B. K., \& Kennedy, D. (2012).
\emph{Calculus: Graphical, Numerical, Algebraic (AP* Edition)}. 4th ed.
Boston, MA: Prentice Hall.

For homework assignments, we will refer to this book as \textbf{Textbook}.

\hypertarget{graphing-calculator}{%
\subsection{Graphing Calculator}\label{graphing-calculator}}

Required: Choose one of the Texas Instruments TI-83/84 series (including TI-83 Plus, TI-84 Plus, TI-84 Silver Edition, TI-84 Plus CE)

I will be using \textbf{TI-84 Plus} for in-class demonstrations, but any one of the TI-83/84 models should be similar enough to follow along in class. Check this \href{https://brownmath.com/ti83/diff8384.htm\#PlusSilver}{link} to see the differences among the TI-83/84 models.

\hypertarget{khan}{%
\subsection{Khan Academy}\label{khan}}

\href{https://www.khanacademy.org/}{Khan Academy Website}

Mr.~Li's Class Code for AP Calculus AB: PHHGRQCY

Instructions:

\begin{enumerate}
\def\labelenumi{\arabic{enumi}.}
\item
  \href{https://www.khanacademy.org/signup}{Create a Khan Academy account} with your school-issued e-mail address.
\item
  Visit \url{https://www.khanacademy.org/join} and enter class code (PHHGRQCY). Alternatively, visit this link (\url{https://www.khanacademy.org/join/PHHGRQCY}) to join the class. Please do not join this Khan Academy classroom unless you are enrolled in Mr.~Li's Calculus AB class.
\end{enumerate}

In traditional classroom formats, teachers lecture on concepts with basic examples and often do not have time to cover difficult problems. We will be using Khan Academy to flip our classroom. That is, most of your homework will involve progressing through Khan Academy modules and practice problems. In class, I will clarify concepts and delve into more challenging problems. Since I will be able to understand each student's individual progress in real time, I will adjust the format of the following day's lesson accordingly. If students are struggling in a unit, we will temporarily revert to the traditional format and review key concepts in class. In cases where students are all on track, we will have more time to engage in harder and nuanced problems as a class and in group work. In other units in which students' progress varies widely, I may use class time for individualized instruction. Based on your feedback and depending on the unit, we will make adjustments along the way.

\hypertarget{recommended-reference}{%
\subsection{Recommended Reference}\label{recommended-reference}}

Bock, D., Donovan, D., \& Hockett, S.O. (2017). \emph{Barron's AP
Calculus}. 4th ed.~Hauppauge, NY: Barron's Educational Series,
Inc.

If you would like more practice and preparation for the AP exam, the text above is an excellent reference. However, you are not required to purchase this text.

\hypertarget{other-resource-links}{%
\subsection{Other Resource Links}\label{other-resource-links}}

\href{https://apcentral.collegeboard.org/courses/ap-calculus-ab/course}{College Board: AP Calculus AB}

\begin{itemize}
\tightlist
\item
  \href{https://apcentral.collegeboard.org/courses/ap-calculus-ab/exam/past-exam-questions}{Past Free Response Questions}
\item
  AP Question Bank (will be released by College Board on August 1)
\end{itemize}

\href{http://tutorial.math.lamar.edu/Classes/CalcI/CalcI.aspx}{Paul's Online Math Notes (Calculus I)}

\hypertarget{ap-exam}{%
\section{AP Exam}\label{ap-exam}}

\href{https://apcentral.collegeboard.org/courses/exam-dates-and-fees/exam-dates-2020}{2020 AP Exam Schedule}

\begin{itemize}
\tightlist
\item
  AP Calculus AB/BC: May 5, 2020 (8 a.m. local time)
\end{itemize}

\href{https://apcentral.collegeboard.org/courses/ap-calculus-ab/exam}{Exam Format}

\begin{itemize}
\tightlist
\item
  Section 1 (50\%): 45 multiple-choice questions (1 hour, 45 min.)

  \begin{itemize}
  \tightlist
  \item
    Non-Calculator Portion: 30 questions (1 hour)
  \item
    Calculator Portion: 15 questions (45 min.)
  \end{itemize}
\item
  Section 2 (50\%): 6 free-response questions

  \begin{itemize}
  \tightlist
  \item
    Calculator Portion: 2 questions (30 min.)
  \item
    Non-Calculator Portion: 4 questions (60 min.)
  \end{itemize}
\end{itemize}

\hypertarget{grades}{%
\section{Grades}\label{grades}}

Final grades for each semester will be calculated as follows:

\begin{itemize}
\tightlist
\item
  20\% Midterm
\item
  20\% Final Exam
\item
  20\% Homework
\item
  10\% Participation
\item
  10\% Free-Response Questions
\item
  10\% First Monthly Exam
\item
  10\% Second Monthly Exam
\end{itemize}

\hypertarget{course-exams}{%
\subsection{Course Exams}\label{course-exams}}

Each semester, the math department will administer two monthly exams, a midterm, and a final exam. Once the exam dates are finalized, they will be listed in this section.

\hypertarget{homework}{%
\subsection{Homework}\label{homework}}

Homework assignments will be listed in the \protect\hyperlink{sch}{Schedule} section under the date that they are assigned. Unless otherwise stated, assignments are due the following school day.

Since I can track your progress on Khan Academy, I will not ask you to turn in assignments from Khan Academy. However, I highly recommend that you work on Khan Academy questions on paper and keep your work organized for exam review.

For textbook questions, I expect the following:

\begin{enumerate}
\def\labelenumi{\arabic{enumi}.}
\item
  Write your name, date, and period on the top-right corner of the first page. For subsequent pages, place just your name (without date/period) on the top-right corner.
\item
  On the first page, write down the assignment as the title. Example: Section 1.1 (pages 10-11, \#1-36 odd)
\item
  For short questions, copy down the entire problem. For longer questions, outline the problem with enough information so that you can understand the question without referring back to the textbook. The point of copying down short problems and outlining longer problems is to allow you to reference your homework problems easily for review and exam preparation.
\end{enumerate}

You will receive full credit for homework as long as you write your own homework solutions and make complete corrections with a pen when we go over solutions in class.

\hypertarget{participation}{%
\subsection{Participation}\label{participation}}

During class, you will have the opportunity to solve problems on the board. This is not only good practice for verbal communication of mathematics but also a great way to solidify your understanding of concepts. After each time at the board, you will sign your name on a sheet so I can tally your participation frequency. As long as you maintain regular participation, you will receive full points in this area.

\hypertarget{free-response-questions}{%
\subsection{Free Response Questions}\label{free-response-questions}}

In class, we will work on previous free response questions that relate to the concepts that we cover in each unit. To simulate the scoring process, I will ask students to exchange free response papers to grade as I go through the solutions and assignment of points. After you receive your scored free response paper, you should use a pen with a different color to correct your mistakes to receive full credit.

\hypertarget{policies}{%
\section{Policies}\label{policies}}

\hypertarget{integrity}{%
\subsection{Integrity}\label{integrity}}

Honesty is the best policy. We will conduct this course on an honor system. This means that we will trust each other to maintain integrity. Please do not cheat or aid others in cheating.

\hypertarget{collaboration-policy}{%
\subsection{Collaboration Policy}\label{collaboration-policy}}

While I highly encourage students to help each other in this course, please observe the following guidelines:

\begin{itemize}
\tightlist
\item
  First try to work on problems on your own. Give yourself time to think through problems.
\item
  If you require additional help, please feel free to work collaboratively with other students on a separate piece of paper. In the process of helping each other, please do not rush through the steps to reach a solution. Allow students struggling with the problem more time to think through each step with minor hints so that they can arrive at the final solution on their own.
\item
  At the end of collaboration, you should still write your own homework solutions from start to finish. Do not short-change your own learning process by copying answers.
\end{itemize}

\hypertarget{curricular-requirements}{%
\section{Curricular Requirements}\label{curricular-requirements}}

\href{https://apcentral.collegeboard.org/pdf/ap-calculus-ab-bc-course-and-exam-description-0.pdf?course=ap-calculus-ab}{College Board AP Calculus AB and BC Course and Exam Description}

Page 17: Exam weighting of each unit for multiple-choice exam
Pages 20--23: Topics for each unit

\begin{itemize}
\tightlist
\item
  In the \protect\hyperlink{sch}{Schedule} section, I will use block quotes to denote the topic we are covering along with the course objectives, which the College Board broadly classifies into \texttt{CHA} (Change), \texttt{LIM} (Limits), and \texttt{FUN} (Analysis of Functions).
\end{itemize}

\hypertarget{timeline}{%
\section{Timeline}\label{timeline}}

\begin{longtable}[]{@{}llr@{}}
\toprule
& Topic & Days\tabularnewline
\midrule
\endhead
September & Limits \& Continuity & 23\tabularnewline
October & Differentiation: Basics & 14\tabularnewline
November & Differentiation: Composite, Implicit, \& Inverse Functions & 9\tabularnewline
December & Differentiation: Contextual Applications & 11\tabularnewline
December & Differentiation: Analytical Applications & 16\tabularnewline
Jan/Feb & Integration & 20\tabularnewline
March & Differential Equations & 9\tabularnewline
Mar/Apr & Integration: Applications & 20\tabularnewline
\bottomrule
\end{longtable}

\hypertarget{sch}{%
\section{Schedule}\label{sch}}

\hypertarget{september}{%
\subsection*{September}\label{september}}
\addcontentsline{toc}{subsection}{September}

\hypertarget{sept.-2}{%
\subsubsection*{Sept.~2}\label{sept.-2}}
\addcontentsline{toc}{subsubsection}{Sept.~2}

\begin{quote}
\textbf{1.1} Introducing Calculus: Can Change Occur at an Instant?
\end{quote}

\begin{quote}
\texttt{CHA-1.A} Interpret the rate of change at an instant in terms of average rates of change over intervals containing that instant.
\end{quote}

\textbf{Homework}

\begin{itemize}
\tightlist
\item
  Follow instructions in Section \ref{khan} to join Mr.~Li's designated Khan Academy classroom for AP Calculus AB.
\item
  Khan Academy:
\item
  Textbook:
\end{itemize}

\hypertarget{sept.-3}{%
\subsubsection*{Sept.~3}\label{sept.-3}}
\addcontentsline{toc}{subsubsection}{Sept.~3}

\hypertarget{october}{%
\subsection*{October}\label{october}}
\addcontentsline{toc}{subsection}{October}

\hypertarget{november}{%
\subsection*{November}\label{november}}
\addcontentsline{toc}{subsection}{November}

\hypertarget{december}{%
\subsection*{December}\label{december}}
\addcontentsline{toc}{subsection}{December}

%manually set hanging indents for references
%\pdfbookmark[0]{\indexname}{Index}
%\printindex


\end{document}
